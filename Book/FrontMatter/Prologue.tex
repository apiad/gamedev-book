\chapter*{Prologue}

The idea for writting this book came out from a course in game development we've being teaching
for a few years now, at the University of Havana. After a few iterations and modifications to
the course syllabus, we (boldly) decided that it would be nice to have the basic content of the
course in a single written material. In this prologue we explain why we think this book is
worth reading, and, if we convince you, then how it is best read.

\section*{Why reading this book?}

The existence of this book is based on two main hypothesis. The first one is concerned with the
content. Most of the books about game development we've seen (and used in our courses) are either
very shallow or too specific. There are many books that don't require almost any background knowledge
and there are others specifically devoted to game AI, physics, rendering, or some other topic.
While we think many of these books are great in their own area, we wanted for our course a book that
was both broad in the topics covered, and deep in the content. So this book is specially designed
for students (or graduates) of a standard Computer Science major. We think that if you know CS,
we can explain many of the concepts in game development in much greater detail, and than that
makes you a better game developer.

The second hypothesis is concerned with the format. Again, many of the books we've seen take one
of two approaches: they either try to be technology-agnostic or they are recipe collections for
a specific technology. In the first group there are many game design books, books on design patterns,
books about psychology and sociology of games, and many other gems. In the second group there
are many great cookbooks, that can give a quick introduction to a specific technology and
make you proficient at it. However, we wanted a hybrid approach for our course. On one hand,
we want to tackle design issues, even psychological and sociological issues, that help
game developers craft more compelling and engaging games. On the other hand, we want to tackle
technological issues, algorithms, design patterns, languages, platforms, devices, which involve
many practical decisions that can make a beatifully designed game either succeed or fail.

For these reasons, the following book is unique (or at least rare) in some senses. First,
it presents theory (both theory of game design as well as theory on algorithms, techniques, etc.)
with the depth and complexity than can be expected in academic rather than comercial books.
Second, it presents plenty of code examples, practical advices, and technical content, oriented
towards a specific game-making technology, which is the Unity game engine, in the extent that
can be expected in comercial rather than academic books. We choose this
particular technology because we believe it has a very well designed API, such that almost all
concepts and patterns fit neatly within its architecture, it is also very efficient, and it has
a thriving community of developers and users that produce plenty of documentation, examples, plugins, and
other resources.

So, in order to answer the original, why should you read this book? Because once finished you will
have a deep theoretical knowledge about how video games are designed and developed, and you will
also have a pretty strong practical experience with the de-facto standard tool for making videogames
in today's industry.

\section*{Who is this book for?}

This book is for people with interest in video games design and development from a theoretical
and practical point of view, with a competent knowledge in Computer Science topics. Particularly,
the book has been designed for undergrad students of a Computer Science major, and we expect you
to be comfortable around the following topics:

\begin{itemize}
	\item Analysis of algorithms, order of growth, computacional complexity.
	\item Computer geometry and related algorithms and data structures.
	\item Numerical analysis and algebra, including numeric integration.
	\item Artificial intelligence, including problem-solving and machine learning.
	\item Formal languages, grammars, regular expressions, automata theory.
	\item Design patterns, basic concepts in software engineering.
	\item Comfortable programming in C\#, or similar languages (Java, C++, \ldots).
\end{itemize}

For all of the above topics, we expect a minimum understanding, at the level that can
be expected in a Computer Science undergrad student that has taken at least the introductory
course on each topic.
That said, there are some chapters in the book that do not require any prior knowledge, concerned
with game mechanics, design, and publishing and marketting.
